\nsection{Car Park}

In a 2-dimesional car park as seen from above, vehicles are allowed to park either vertically or horizontally on a grid~:
\renewcommand{\A}{|[fill=gray,text=red]|A}
\renewcommand{\B}{|[fill=gray,text=green]|B}
\renewcommand{\X}{|[fill=ocre,text=ocre]|X}
\renewcommand{\H}{|[fill=gray,text=gray]|.}

\begin{tikzpicture}[every node/.style={anchor=base,text depth=.5ex,text height=2ex,text width=1em,outer sep=0pt,align=center,inner sep=0pt}]
\matrix [matrix of nodes,draw=white,nodes in empty cells]
{
\X&\H&\X&\X&\X&\X\\
\H&\B&\B&\B&\H&\X\\
\X&\A&\H&\H&\H&\X\\
\X&\A&\H&\H&\H&\X\\
\X&\A&\H&\H&\H&\X\\
\X&\X&\X&\X&\X&\X\\
};
\end{tikzpicture}

Here there are two vehicles, both of length $3$. One is vertical (aligned
north/south) coloured red, and one is horizontal (aligned west/east)
coloured green. Empty cells of the gridded car park are shown in gray,
and immovable boundaries (bollards) are shown in orange. Vehicles can
only move forwards or backwards one square at a time (vertical vehicles
move up and down, horizontal vehicles move left and right). These moves
cannot be into, or across, another vehicle, nor into a bollard.

Our aim is to get all the vehicles safely out of the car park.
During a turn, one vehicle is allowed to move forwards or backwards one
square.  Once the front of a vehicle touches an empty square on the edge
of the car park, it is deemed to have exited the car park, and removed.

To `solve' the car park above, we would move vehicle B one move left (so that it exits)~:

\begin{tikzpicture}[every node/.style={anchor=base,text depth=.5ex,text height=2ex,text width=1em,outer sep=0pt,align=center,inner sep=0pt}]
\matrix [matrix of nodes,draw=white,nodes in empty cells]
{
\X&\H&\X&\X&\X&\X\\
\H&\H&\H&\H&\H&\X\\
\X&\A&\H&\H&\H&\X\\
\X&\A&\H&\H&\H&\X\\
\X&\A&\H&\H&\H&\X\\
\X&\X&\X&\X&\X&\X\\
};
\end{tikzpicture}

and then vehicle A up one~:

\begin{tikzpicture}[every node/.style={anchor=base,text depth=.5ex,text height=2ex,text width=1em,outer sep=0pt,align=center,inner sep=0pt}]
\matrix [matrix of nodes,draw=white,nodes in empty cells]
{
\X&\H&\X&\X&\X&\X\\
\H&\A&\H&\H&\H&\X\\
\X&\A&\H&\H&\H&\X\\
\X&\A&\H&\H&\H&\X\\
\X&\H&\H&\H&\H&\X\\
\X&\X&\X&\X&\X&\X\\
};
\end{tikzpicture}

and then up one more move so that it too exits~:

\begin{tikzpicture}[every node/.style={anchor=base,text depth=.5ex,text height=2ex,text width=1em,outer sep=0pt,align=center,inner sep=0pt}]
\matrix [matrix of nodes,draw=white,nodes in empty cells]
{
\X&\H&\X&\X&\X&\X\\
\H&\H&\H&\H&\H&\X\\
\X&\H&\H&\H&\H&\X\\
\X&\H&\H&\H&\H&\X\\
\X&\H&\H&\H&\H&\X\\
\X&\X&\X&\X&\X&\X\\
};
\end{tikzpicture}

All vehicles have now exited the car park taking a total of three turns.


\begin{exercise}
Write a program that reads in a car park file (specified on the command line), and shows the `turns' to solve it. The file for the car park above looks like~:
\begin{terminaloutput}
6x6
#.####
.BBB.#
#A...#
#A...#
#A...#
######
\end{terminaloutput}

\noindent The first line has two numbers; the height of
the car park (number of rows) and then the width (number of columns).

\noindent
In the remainder of the file, vehicles are shown as a capital letter,
gaps as a full-stop and bollards as a hash symbol. Each cars may only
lie in the grid vertically or horizontally, and must be of at least 
length $2$. Each vehicle must have a unique uppercase letter, the first of which must be an `A', the next one be `B' and so on.

\noindent
We wil use a brute-force algorithm for searching over all moves for a
solution~:
\begin{enumerate}
\item You will use an array (list) of structures, each one containing the data for one car park.
Note that you may choose to store the state of each car park, not as
a 2D array, but as something that is easier to manipulate, e.g. an array
of vehicles, including their position, orientation and whether they've exited or not.
Each approach has pros and cons.
\item Put the initial car park into the front of this list, \verb^f=0^.
\item Consider the car park at the {\bf front} of the list (index \verb$f$).
\item For this (parent) car park, find the resulting (child) car parks 
which can be created from all the possible vehicle moves. For each of these child car parks:
\begin{itemize}
\item If this car park is unique (i.e.\ it has not been seen before in the list), add it to the end of the list.
\item If it has been seen before (a duplicate) ignore it.
\item If it is the `final' car park, stop and print the solution.
\end{itemize}
\item Add one to $f$. If there are more car parks in the list, go to step $3$.
\end{enumerate}

\noindent To help with printing out the correct moves, when a solution
has been found, each structure in the list will need to contain (amongst
other things) a car park and a record of its parent car park, i.e. the car park
that it was created from. Since you're using an array, this could simply
be which index of the array was the parent.

\noindent The program reads the name of the car park definition file from
the command line.  If it finds a successful solution, it prints out the
number of car parks that would be printed in the solution and {\bf nothing
else}, or else exactly the phrase `No Solution?'' if none can be found
(as might be the case if it is impossible, or you simply run out of memory)~:

\begin{terminaloutput}
$ ./carpark ../Git/Data/CarPark/6x6_2c_3t.prk
3 moves
$ .car/park ../Git/Data/CarPark/11x9_10c_26t.prk
26 moves
\end{terminaloutput}

If the `show' flag is used, your program will print out the solution in the correct order~:
\begin{terminaloutput}
$ ./carpark -show ../Git/Data/CarPark/6x6_2c_3t.prk
#.####
.BBB.#
#A...#
#A...#
#A...#
######

#.####
.....#
#A...#
#A...#
#A...#
######

#.####
.A...#
#A...#
#A...#
#....#
######

#.####
.....#
#....#
#....#
#....#
######

3 moves
\end{terminaloutput}

\noindent
Your program~:
\begin{itemize}
\item {\bf Must} use the algorithm detailed above (which is similar to a queue and therefore a breadth-first search). Do not use the other algorithms possible (e.g. best-first, guided, recursive etc.); the quality of your coding is being assessed against others taking the same approach.
\item {\bf Must not} use dynamic arrays or linked lists. Since car parks cannot be any larger than $20 \times 20$, you can create car parks of this size, and only use a sub-part of them if the car park required is smaller. The list of car parks can be a fixed (large) size.
\item {\bf Should} check for invalid car park definition files, and report in a graceful way if there is a problem, aborting with \verb^exit(EXIT_FAILURE)^ if so.
\item {\bf Should not} print anything else out to screen after successfully
completing the search, except that which is shown above. Automated checking
will be used during marking, and therefore the output must be precise.
\item {\bf Should} call the function \verb^test()^ to perform any assertion testing etc.
\end{itemize}

\subsection*{Extension}

Basic assignment = {\Large $90\%$}.
Extension = {\Large $10\%$}.

\noindent
If you'd like to try an extension, make sure to submit {\it extension.c}
and a brief description in a {\it extension.txt} file, and an {\it extension.mak} Makefile, allowing me to build your code using \verb^make extension^.
\noindent The extension could
involve a faster search technique, better graphical display, user input
or something else of your choosing.
\end{exercise}
