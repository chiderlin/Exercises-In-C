\nsection{Diffusion Limited Aggregation}

\tikzset{twocolour board/.style={
  matrix of nodes
  , execute at empty cell={\node[text=white,fill=white]{+};}
  , nodes in empty cells=false
  , nodes={draw=gray,fill=ocre,minimum width=#1,minimum height=#1,outer sep=0pt,align=center,inner sep=0pt,font=\tiny}
  , text=ocre
  , row sep={#1,between origins}
  , column sep={#1,between origins}
},
  twocolour board/.default=0.2cm
}

\wwwurl{https://en.wikipedia.org/wiki/Diffusion-limited_aggregation}

\begin{quote}
In its simplest form, DLA occurs on a grid of square
cells.  The cell at the center of the grid is the location of the
seed point, a particle stuck at that square.  Now pick a square on the
perimeter of the grid and place a wandering particle on that square.
At each iteration, this particle moves to one of the four
adjacent squares, left, right, above, or below.
When a wandering particle arrives at one of the four squares adjacent
to the seed, it sticks there forming a cluster of two particles, and
another edge particle is released.  When a moving particle arrives at
one of the squares adjacent to the cluster, it sticks there.
\end{quote}

\begin{center}
\begin{tikzpicture}
\matrix [twocolour board]
{
&&&&&&&&&&&&&&&&&&&&&&&&&&&&&&&&&&&&&&&&&&&&&&&&&&\\
&&&&&&&&&&&&&&&&&&&&&&&@&&&&&&&&&&&&&&&&&&&&&&&&&&&\\
&&&&&&&&&&&&&&&&&&&&&&@&@&@&@&@&@&&&&&&&&&&&&&&&&&&&&&&&\\
&&&&&&&&&&&&&&&&&&&&&&&&&@&&&&&&&&&&&&&&&&&&&&&&&&&\\
&&&&&&&&&&&&&&&&&&&&&&&&&@&&&&@&&&&&&&&&&&&&&&&&&&&&\\
&&&&&&&&&&&&&&&&&&&&&&&&&@&@&@&&@&@&&&&&&&&&&&&&&&&&&&&\\
&&&&&&&&&&&&&&&&&&&&&@&&&&&@&@&@&@&&&&&&&&&&&&&&&&&&&&&\\
&&&&&&&&&&&&&&&&&&&&@&@&&&&&&@&&&&@&&&&&&&&&&&&&&&&&&&\\
&&&&&&&&&&&&&&&&&&&&@&@&&&&&&@&&@&@&@&&&&&&&&&&&&&&&&&&&\\
&&&&&&&&&&&&&&&&&&&&&@&&&&&@&@&&@&&&&&&&&&&&&&&&&&&&&&\\
&&&&&&&&&&&&&&&&&&&&&@&@&@&&&&@&@&@&@&&&&&&&&&&&&&&&&&&&&\\
&&&&&&&&&&&&&&&&&&&&&&&@&@&@&&@&@&&&&@&&@&&&&&&&&&&&&&&&&\\
&&&&&&&&&&&&&&&&&&&&@&@&@&@&&@&@&@&@&@&@&@&@&@&@&&&&&&&&&&&&&&&&\\
&&&&&&&&&&&&&&&&&&&&&&&&&&@&&&&@&&@&&&&&&&&&&&&&&&&&&\\
&&&&&&&&&&&&&&&&&&&&&&&&&&@&&&&&&@&&&&&&&&&&&&&&&&&&\\
&&&&&&&&&&&&@&&&&&&&@&@&&&&&&@&@&&&&&&&&&&&&&&&&&&&&&&&\\
&&&&&&&&&&&&@&&&&&&&&@&&@&&@&&@&&&&&&&&&&&&&&&&&&&&&&&&\\
&&&&&&&&&&&@&@&&&&&&@&@&@&@&@&@&@&@&@&@&&&&&&&&&&&&&&&&&&&&&&&\\
&&&&&&&&&&&&@&@&&&&&&&&&&&&@&@&&&&&&&&&&&&&&&&&&&&&&&&\\
&&&&&&&&&&&&&@&&@&@&@&@&@&@&@&&&@&@&@&&&&&&&&&&&&&&&&&&&&&&&&\\
&&&&&&&&&@&@&@&@&@&@&@&@&@&&@&@&@&@&@&@&@&@&@&@&@&@&@&&&&&&&&&&&&&&&&&&&\\
&&&&&&&&@&@&&@&&@&&&&@&&&&&&&&@&&&&&@&&&&&&&&&&&&&&&&&&&&\\
&&&&&&&&&&&&&&&&@&@&&&&&&&&@&@&&&&&&&&&&&&&&&&&&&&&&&&\\
&&&&&&&&&&&&&&&&&&&&&&&&&@&&&&&&&@&&&&&&&&&&&&&&&&&&\\
&&&&&&&&&&&&&&&&&&&&&&&&&@&&&&&&@&@&&&&&&&&&&&&&&&&&&\\
&&&&&&&&&&&&&&&&&&&&@&&&&@&@&&&@&@&@&@&&&&&&&&&&&&&&&&&&&\\
&&&&&&&&&&&&&&&&@&&&&@&@&@&@&@&@&@&@&@&@&&@&&&&&&@&@&&&&&&&&&&&&\\
&&&&&&&&&&&&&&&&@&@&@&@&@&@&&&@&&&@&&&&&&&&&&@&@&&&&&&&&&&&&\\
&&&&&&&&&&&&&&&&&&@&&&&&@&@&@&&@&@&&&&&&&&@&@&&&&&&&&&&&&&\\
&&&&&&&&&&&&&&&&&&@&&&&&@&&@&&&@&@&&&@&&&&@&&&&&&&&&&&&&&\\
&&&&&&&&&&&&&&&@&@&@&@&&&&&&&@&&&@&@&&&@&&&@&@&&&&&&&&&&&&&&\\
&&&&&&&&&&&&&&&&&&@&&&&&&&&&&@&@&@&@&@&@&&@&@&&@&&@&&&&&&&&&&\\
&&&&&&&&&&&&&&&&&@&@&@&&&&&&&&&@&&&@&&@&@&@&@&@&@&@&@&@&&&&&&&&&\\
&&&&&&&&&&&&&&&&&&@&@&&&&&&&&&@&@&@&@&&&@&@&&@&&@&&@&@&@&&&&&&&\\
&&&&&&&&&&&&&&&&&&&@&&&&&&&@&@&@&&@&&&&@&&&@&&@&&@&@&@&@&&&&&&\\
&&&&&&&&&&&&&&&&&&&@&&&&&&&&&&&&&&&&&&&&&&&@&&&&&&&&\\
&&&&&&&&&&&&&&&&&&&&&&&&&&&&&&&&&&&&&&&&&&@&&&&&&&&\\
&&&&&&&&&&&&&&&&&&&&&&&&&&&&&&&&&&&&&&&&&&&&&&&&&&\\
};
\end{tikzpicture}
\end{center}

\noindent Note that the grid is toroidal - this means that if a particle goes off the top of the grid, it reappears at the bottom. Likewise, if it goes off the edge, it will appear on the other side.

\begin{exercise}
\label{ex:dla}
Using the algorithm outlined above, using a $50 \times 50$ board,
output each of $250$ iterations
(here iteration means when each particle has become `stuck').
The output should be in plain text.
\end{exercise}


In the basic version of a DLA, a particle `sticks' to the seed points when
it gets adjacent to one. Here we introduce the probability of stickiness,
a number between $0$ and $1$, called $p_s$. The particle only sticks
to a seed point with a probability of $p_s$; if
it doesn't stick it continues wandering. This allows for the patterns
that emerge to be more hairy and solid:
\wwwurl{http://paulbourke.net/fractals/dla/}

\begin{exercise}
Implement this stickiness concept by extending the program written for Exercise~\ref{ex:dla}, allowing the user to specify a `stickiness' probability via the command line using argv[1]. Here a value of $1.0$ means that the particle will always stick, and $0.5$ means that it will stick one time in $2$.
\end{exercise}
\noindent Here's an example when setting $p_s=0.25$:
\begin{center}
\begin{tikzpicture}
\matrix [twocolour board]
{
&&&&&&&&&&&&&&&&&&&&&&&&&&&&&&&&&&&&&&&&&&&&&&&&&&\\
&&&&&&&&&&&&&&&&&&&&&&&&&&&&&&&&&&&&&&&&&&&&&&&&&&\\
&&&&&&&&&&&&&&&&&&&&&&&&&&&&&&&&&&&&&&&&&&&&&&&&&&\\
&&&&&&&&&&&&&&&&&&&&&&&&&&&&&&&&&&&&&&&&&&&&&&&&&&\\
&&&&&&&&&&&&&&&&&&&&&&&&&&&&&&&&&&&&&&&&&&&&&&&&&&\\
&&&&&&&&&&&&&&&&&&&&&&&&&&&&&&&&&&&&&&&&&&&&&&&&&&\\
&&&&&&&&&&&&&&&&&&&&&&&&&&&&&&&&&&&&&&&&&&&&&&&&&&\\
&&&&&&&&&&&&&&&&&&&&&&&&&&&&&&&&&&&&&&&&&&&&&&&&&&\\
&&&&&&&&&&&&&&&&&&&&&&&&&&&&&&&&&&&&&&&&&&&&&&&&&&\\
&&&&&&&&&&&&&&&&&&&&&&&&&&&&&&&&&&&&&&&&&&&&&&&&&&\\
&&&&&&&&&&&&&&&&&&&&&&&&&&&&&&&&&&&&&&&&&&&&&&&&&&\\
&&&&&&&&&&&&&&&&&&&&&&&&&&&&&&&&&&&&&&&&&&&&&&&&&&\\
&&&&&&&&&&&&&&&&&&&&&&&&&&&&&&&&&&&&&&&&&&&&&&&&&&\\
&&&&&&&&&&&&&&&&&&&&&&&&&&&&&&&&&&&&&&&&&&&&&&&&&&\\
&&&&&&&&&&&&&&&&&&&&&&&&&&&&&&&&&&&&&&&&&&&&&&&&&&\\
&&&&&&&&&&&&&&&&&&&&&&&&&&&&&&@&@&&&&&&&&&&&&&&&&&&&\\
&&&&&&&&&&&&&&&&&&&&&&&&@&&@&&@&@&@&@&@&&&@&&&&&&&&&&&&&&&\\
&&&&&&&&&&&&&&&&&&&&&&&&@&@&@&&&&@&@&&&&@&@&@&&&&&&&&&&&&&\\
&&&&&&&&&&&&&&&&&&&&&&&&&&@&@&&&@&@&@&@&@&@&@&&&&&&&&&&&&&&\\
&&&&&&&&&&&&&&&&&&&&&&&&&&@&@&&@&@&@&@&&@&&@&&&&&&&&&&&&&&\\
&&&&&&&&&&&&&&&&&&&&&&&&&&@&@&@&@&&@&&&&&&&&&&&&&&&&&&&\\
&&&&&&&&&&&&&&&&&&&&&&@&@&&&@&@&&&&&&&@&&&&&&&&&&&&&&&&\\
&&&&&&&&&&&&&&&&&&&&&&@&@&@&&@&@&@&&&&@&&@&&&&&&&&&&&&&&&&\\
&&&&&&&&&&&&&&&&&&&&&&@&@&@&&@&&@&@&@&@&@&@&@&&&&@&&&&&&&&&&&&\\
&&&&&&&&&&&&&&&&&&&&&@&@&@&@&@&@&@&@&&&&&&@&&&@&@&&&&&&&&&&&&\\
&&&&&&&&&&&&&&&&&&&&&@&@&@&@&@&&&&&&&&&@&@&@&@&&&&&&&&&&&&&\\
&&&&&&&&&&&&&&&&&&&&&&&&@&@&&&@&&&&&&@&&&&&&&&&&&&&&&&\\
&&&&&&&&&&&&&&&&&&&&&&&&&@&&&@&@&&&&&@&@&&&&&&&&&&&&&&&\\
&&&&&&&&&&&&&&&&&&@&@&@&&&&&@&&@&@&&&&&&&&&&&&&&&&&&&&&&\\
&&&&&&&&&&&&&&&&&&@&@&@&@&@&@&@&@&@&@&@&&&&&&&&&&&&&&&&&&&&&&\\
&&&&&&&&&&&&&&&&&&&&@&&@&@&@&@&&&&&&&&&&&&&&&&&&&&&&&&&\\
&&&&&&&&&&&&&&&@&&&&@&@&@&&@&&@&@&@&@&&&&&&&&&&&&&&&&&&&&&&\\
&&&&&&&&&&&&&&&@&&&&@&@&@&@&@&&@&@&&&&&&&&&&&&&&&&&&&&&&&&\\
&&&&&&&&&&&&&&@&@&&&@&@&@&@&@&&&&@&@&&&&&&&&&&&&&&&&&&&&&&&\\
&&&&&&&&&&&&@&@&@&@&@&@&@&@&&@&&&&&@&@&&&&&&&&&&&&&&&&&&&&&&&\\
&&&&&&&&&&&&&@&&&@&@&&&@&@&&&&&&@&@&&&&&&&&&&&&&&&&&&&&&&\\
&&&&&&&&&&&&&&&&@&&&&&@&@&&&&@&@&@&@&&&&&&&&&&&&&&&&&&&&&\\
&&&&&&&&&&&&&&&&@&&&&&@&&&&@&@&&@&@&&&&&&&&&&&&&&&&&&&&&\\
&&&&&&&&&&&&&&&&&&&&&&&&&@&&&@&@&&&&&&&&&&&&&&&&&&&&&\\
&&&&&&&&&&&&&&&&&&&&&&&&&@&&&&@&@&&&&&&&&&&&&&&&&&&&&\\
&&&&&&&&&&&&&&&&&&&&&&&&&&&&&&@&@&@&&&&&&&&&&&&&&&&&&\\
&&&&&&&&&&&&&&&&&&&&&&&&&&&&&&&@&&&&&&&&&&&&&&&&&&&\\
&&&&&&&&&&&&&&&&&&&&&&&&&&&&&&&&&&&&&&&&&&&&&&&&&&\\
&&&&&&&&&&&&&&&&&&&&&&&&&&&&&&&&&&&&&&&&&&&&&&&&&&\\
&&&&&&&&&&&&&&&&&&&&&&&&&&&&&&&&&&&&&&&&&&&&&&&&&&\\
&&&&&&&&&&&&&&&&&&&&&&&&&&&&&&&&&&&&&&&&&&&&&&&&&&\\
&&&&&&&&&&&&&&&&&&&&&&&&&&&&&&&&&&&&&&&&&&&&&&&&&&\\
&&&&&&&&&&&&&&&&&&&&&&&&&&&&&&&&&&&&&&&&&&&&&&&&&&\\
&&&&&&&&&&&&&&&&&&&&&&&&&&&&&&&&&&&&&&&&&&&&&&&&&&\\
&&&&&&&&&&&&&&&&&&&&&&&&&&&&&&&&&&&&&&&&&&&&&&&&&&\\
};
\end{tikzpicture}
\end{center}

Using different planar walks. e.g. in base-4:
\verb^http://demonstrations.wolfram.com/UsingIrrationalSquareRootsToCreatePlanarWalks/^
