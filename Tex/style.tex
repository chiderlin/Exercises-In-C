\nsection{Correctness}

These style rules ensure your code is as-correct-as-can-be with the aid
of the compiler and other tools:
\begin{description}
\item[FLAGS] Having no warnings (or errors!) when compiling and executing with the flags:\\
For array bounds checking, \verb^NULL^ pointers being dereferenced etc:
\begin{terminaloutput}
-Wall -Wextra -Wfloat-equal -Wvla -pedantic -std=c99
-fsanitize=undefined -fsanitize=address -g3
\end{terminaloutput}
For memory leaks:
\begin{terminaloutput}
-Wall -Wextra -Wfloat-equal -Wvla -pedantic -std=c99
-g3
\end{terminaloutput}
then run:
\begin{terminaloutput}
valgrind --leak-check=full ./myexec
\end{terminaloutput}
For `final' production-ready code:
\begin{terminaloutput}
-Wall -Wextra -Wfloat-equal -Wvla -pedantic -std=c99
-O3
\end{terminaloutput}

You can use more flags than this, obviously, but these will
make sure a few of the essential warnings that commonly indicate
the presence of bugs and leaks are checked. These guidelines are meant to
be independent of the particular compiler used though. Sometimes it is helpful to use many compilers too, e.g. \verb^gcc^ and \verb^clang^.

If you have unused variables (for example) in your code, it doesn't matter whether your compiler happened to tell you about it or not - it's still wrong~!

\item[BRACE] Always brace all functions, \verb^for^s, \verb^while^s, \verb^if/else^ etc.
Somewhat controversial, this ensures that `extra' lines tagged onto loops are
dealt with correctly. For instance:
\begin{codesnippet}
while(i < 10)
   printf("%i\n", i);
   i++;
\end{codesnippet}
looks like it should print out \verb^i^ $10$ times, but instead runs infinitely.
The programmer probably meant:
\begin{codesnippet}
while(i < 10){
   printf("%i\n", i);
   i++;
}
\end{codesnippet}

\item[GOTO] You do not use any of the keywords \verb^continue^, \verb^goto^ or \verb^break^. The one exception is inside \verb^switch^, where \verb^break^ is allowed because it is essential~!
These keywords usually lead to tangled and complex `spaghetti' coding style.
I often recommend that you rewrite the offending code using functions, which {\bf can} have multiple \verb^return^s
in them.

\item[NAMES] Meaningful identifiers. Make sure that functions names and variables having meaningful, but succinct,  names.

\item[REPC] Repetitive code. If you've cut-and-paste large chunks of code, and made minor changes to it, you've done it wrong. Make it a function, and pass parameters that make the changes required.
\begin{codesnippet}
int inbounds1(int i){
   if(i >=0 && i < MAX){
      return 1;
   }
   else{ 
      return 0;
   }
}

int inbounds2(int i){
   if(i >=0 && i < LEN){
      return 1;
   }
   else{ 
      return 0;
   }
}
\end{codesnippet}
might make more sense as:
\begin{codesnippet}
int inbounds2(int i, int mx){
   if(i >=0 && i < mx){
      return 1;
   }
   else{ 
      return 0;
   }
}
\end{codesnippet}

\item[GLOB] No global variables. Global variables are declared `above' \verb^main()^, are in scope of all functions, and can be altered {\bf anywhere} in the code. This makes it rather unclear {\bf which} functions should be reading or writing them. You can make a case for saying that occasionally they could be useful (or better) than the alternatives, but for now, they are banned~!

\item[RETV] Any functions that returns a value, should have it used:
\begin{codesnippet}
scanf("%i", &i);
\end{codesnippet}
is incorrect. It returns a value that is ignored. Instead do:
\begin{codesnippet}
	if(scanf("%i", &i) != 1{
	   /* PANIC */
\end{codesnippet}

The only exceptions are \verb^printf^ and \verb^putchar^ which do return values but
which are typically ignored.

\item[MATCH] For every \verb^fopen^ there should be a matching \verb^fclose^.
For every \verb^malloc^ there should be a \verb^free^.
This helps avoid memory leaks, when your program or functions are later used
in a larger project.

\item[STDERR] When exiting your program in an error state, make sure that you \verb^fprintf^ the error on \verb^stderr^ and not \verb^stdout^. Use \verb^exit^, e.g.
\begin{codesnippet}
if(argc != 2){
   fprintf(stderr, "Usage : %s <filename>\n", argv[0]);
   exit(EXIT_FAILURE);
}
\end{codesnippet}


\end{description}

\nsection{Prettifying}

These rules are about making your code easier to read and having a consistent
style in a form that others are expecting to see.

\begin{description}

\item[LLEN] Line length. Many people use terminal and editors that are
of a fixed-width. Having excessively long lines may cause the viewer to scroll to
off the screen. Keep lines short, perhaps $< 60$ characters. However, in a similar way
to the {\bf FLEN} rule below, it's really about the complexity of the line that's the issue,
not its absolute length. A programmer would generally find:
{\small
\begin{codesnippet}
bool arrcleanse(cell oldarr[HEIGHT][WIDTH], cell newarr[HEIGHT][WIDTH], int h, int w)
\end{codesnippet}
}
a great deal easier to read than:
\begin{codesnippet}
if(a < b && j++ >= szpar(e ? true : false) || h==4){
\end{codesnippet}
despite it being twice as long.

\item[TABS] Don't use tabs to indent your code. Every editor views these differently, so you have no guarantee that I'm seeing the same layout as you do. Use spaces. This also prevents issues when cutting-and-pasting from one source to another.

\item[INDENT] Indentation: choose a style for indentation
and keep to it. I happen to use $3$ spaces, put opening braces
for functions on a new line, but at the end of \verb^if^,\verb^else^, \verb^for^, \verb^while^ etc, then close them on a new line, underneath the `i' of the \verb^if^:
\begin{codesnippet}
int smallest(int a, int b)
{
   if(a < b){
      return a;
   }
   else{
      return b;
   }
}
\end{codesnippet}
You can use any style you like, as long as it's consistent.

\item[MAIN] The code should have function prototypes/definitions first, then \verb^main()^, followed by the function implementation. This means the reader always know where to find \verb^main()^, since it will be near the top of the file.


\item[CAPS] Constants are \verb^#define^d, and use all CAPITALS. For instance:
\begin{codesnippet}
#define WEEKS 52
#define MAX(a,b) (a < b ? b:a)
\end{codesnippet}


\item[FLEN] Short functions. All functions are short. It's quite
difficult to put a maximum number of lines on this, but use $20$ as a
starting point. Exceptions include a function that simply prints a list
of instructions. There would be no benefit in splitting it into smaller
functions. Short functions are easier to plan, write and test.

I find it more useful to think about how hard the function is to
understand, rather than its length. Therefore, a $30$ line, simple
function is fine, but an extremely complex and dense $15$ line function
might need to be split up, or more self-documentation added.


\end{description}

\nsection{Readability}

Your code should be self-documenting. Comments will be written when there
is something complex to explain, and only read when something has gone
catastrophically wrong. In many cases clever use of coding will avoid the
need for them.  The compiler never sees them, so cannot check them. If you
change your code, but not your comments, they can be highly misleading.

As Kevlin Henney said~:
\begin{quote}
A common fallacy is to assume authors of incomprehensible code will somehow
be able to express themselves lucidly and clearly in comments. 
\end{quote}

\begin{description}
\item[MAGIC] No magic numbers. There should be no inexplicable numbers in your code, such
as:
\begin{codesnippet}
	if(i < 36){
\end{codesnippet}

It's probably unclear to the reader where the $36$ has come from, or what it means,
even if it is obvious to the programmer at the time of writing the code. Instead,
\verb^#define^ them with a meaningful name.
Array overruns are often cured by being consistent with \verb^#define^s.

\item[BRIEF] Comments are brief, and non-trivial. Worthless commenting often
looks something like:
\begin{codesnippet}
// Set the variable i to zero
int i = 0;
\end{codesnippet}
The programmer extracts no additional information from it. However, for more
difficult edge cases, a comment might be useful.
\begin{codesnippet}
// Have we reached the end of the list ?
if(t1->h == NULL){ 
\end{codesnippet}

\noindent To prevent lines from becoming too long, it is good practice to put comments above
the line it refers to, not at the end of the same line.

\item[TYPE] You should use \verb^typedef^s, \verb^enum^s and \verb^struct^s to
increase readability. 

\item[INFIN] No loops should be infinite. I'll never ask you to write a program that is meant to run forever. Therefore statements such as
\begin{codesnippet}
while(1){
\end{codesnippet}
or
\begin{codesnippet}
for(;;;){
\end{codesnippet}
are to be avoided.

\item[2DINDEX] 2D Arrays in C are indexed \verb^[row][col]^.  Sometimes
it may still work correctly, especially if you've consistently confused
the two.  Therefore, if you write code that indexes it \verb^[col][row]^,
or \verb^[x][y]^ it will confuse anyone else trying to understand (or
reuse) your code. If you were to sketch a graph using \verb^(i,j)^ you'd
almost certainly make $i$ the horizontal axis, and $j$ the vertical.
Therefore, for any two variables it makes more sense to write \verb^[b][a]^
or \verb^[j][i]^.

\end{description}
